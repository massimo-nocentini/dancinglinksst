
\documentclass{beamer}

\usepackage{euler}
\usepackage[utf8]{inputenc}
\usepackage[T1]{fontenc}
\usepackage{textcomp}
\usepackage{minted}
\usepackage[scaled=0.8]{beramono}
\usepackage{CJKutf8}
\usepackage{amsmath}
\usepackage{amsthm}
\usepackage{amssymb}

\usefonttheme[onlymath]{serif}

\usemintedstyle{friendly}

\setbeamertemplate{blocks}[rounded]

%Information to be included in the title page:
\title{Dancing Links\\\small{an educational pearl}}
\author{Massimo Nocentini}
\institute{University of Florence, Italy}
\date{ESUG2019 }

\begin{document}

\frame{\titlepage}

\begin{frame}[fragile]
\frametitle{}
\begin{minted}[fontsize=\small]{smalltalk}
outline
    ^ LinkedList new
        add: 'me and motivations';
        add: 'exact cover problem';
        add: 'DoubleLink objs';
        add: 'Dancing, Dancing...';
        add: 'and even more Dancing!';
        add: 'N-Queens and Sudoku problems';
        yourself
\end{minted}
\end{frame}

\begin{frame}[fragile]
\frametitle{Hi!}
\begin{Verbatim}[fontsize=\small]
$ whoami
Massimo Nocentini, PhD
Mathematician (algebraic combinatorics, formal methods for algs)
Programmer (automated reasoning, logics and symbolic comp)
https://github.com/massimo-nocentini/dancinglinksst
\end{Verbatim}
\vfill
In Donald's words\footnote{\url{https://arxiv.org/abs/cs/0011047}}:
\emph{
Suppose $x$ points to an element of a doubly linked list;
let $L[x]$ and $R[x]$ point to the predecessor and successor
of that element. Then:
\begin{displaymath}
  L[R[x]] \leftarrow L[x],\quad R[L[x]] \leftarrow R[x] \quad(1)
\end{displaymath}
remove $x$ from the list; every programmer knows this.
But comparatively few programmers have realized that
\begin{displaymath}
  L[R[x]] \leftarrow x,\quad R[L[x]] \leftarrow x \quad(2)
\end{displaymath}
will put $x$ back again, with no refs to the whole list at all.
}

\end{frame}

\begin{frame}[fragile]
\frametitle{Main ideas}

\begin{itemize}
  \item Operation $(2)$ arises in \textit{backtrack programs}, which enumerate all 
  solutions to a given set of constraints and it was introduced in 1979 by Hitotumatu and Noshita.
  \item The beauty of $(2)$ is that operation $(1)$ can be undone by knowing only the value of $x$.
  \item We can apply $(1)$ and $(2)$ repeatedly in complex data structures that involve large 
  numbers of interacting doubly linked lists.
  \item This process causes the pointer variables inside the global data structure to execute an 
  exquisitely choreographed dance; hence I like to call $(1)$ and $(2)$ the \textit{technique of dancing links}.
  \item Minato et al. \footnote{\url{https://aaai.org/ocs/index.php/AAAI/AAAI17/paper/view/14907}} 
  constructs a Zero-suppressed BDD (ZDD) that represents the set of sols and it enables the efficient 
  use of memo cache to speed up the search.
\end{itemize}

\end{frame}


\begin{frame}[fragile]
\frametitle{\texttt{DoubleLink}s and circular \texttt{DoubleLinkedList}s}
\texttt{DoubleLink} objects respond to messages
\begin{minted}[fontsize=\footnotesize]{smalltalk}
remove
  nextLink ifNotNil: [ :next | next previousLink: previousLink ].
  previousLink ifNotNil: [ :previous | previous nextLink: nextLink ]
\end{minted}
and
\begin{minted}[fontsize=\footnotesize]{smalltalk}
restore
  nextLink ifNotNil: [ :next | next previousLink: self ].
  previousLink ifNotNil: [ :previous | previous nextLink: self ]
\end{minted}
that implement operations $(1)$ and $(2)$, respectively;
moreover, we extend \texttt{DoubleLinkedList} objects with the message
\begin{minted}[fontsize=\footnotesize]{smalltalk}
makeCircular
  head
    ifNotNil: [ 
      head previousLink: tail.
      tail nextLink: head ]
\end{minted}
to introduce circular lists, doubly connected.
\end{frame}

\begin{frame}[fragile]
\frametitle{AlgorithmX instance side}
\begin{minted}[fontsize=\footnotesize]{smalltalk}
searchDepth: k forDLRootObject: h partialSelection: cont
  ^ (h isFixPointOf: [ :ro | ro right ])
      ifTrue: [ self yieldNode: top onBlock: cont ]
      ifFalse: [ 
        memo
          at: h columns
          ifPresent: [ :tree | self yieldNode: tree onBlock: cont ]
          ifAbsentPut: [ 
            self
              searchDepth: k
              forDLColumnObject: h chooseColumn
              partialSelection: cont ] ]
\end{minted}
\end{frame}

\begin{frame}[fragile]
\frametitle{AlgorithmX instance side}
\begin{minted}[fontsize=\footnotesize]{smalltalk}
searchDepth: k forDLColumnObject: c partialSelection: sel
  ^ self
      onEnter: [ c cover ]
      do: [ 
        c
          untilFixPointOf: [ :co | co up ]
          foldr: [ :r :x | 
            | y |
            y := self searchDepth: k 
                      forDLDataObject: r 
                      partialSelection: sel.
            y isZDDBottom
                ifTrue: [ x ]
                ifFalse: [ 
                  self
                    uniqueNodeWithDLDataObject: r
                    withLowerNode: x
                    withHigherNode: y ] ]
          init: bottom ]
      onExit: [ c uncover ]
\end{minted}
\end{frame}

\begin{frame}[fragile]
\frametitle{AlgorithmX instance side}
\begin{minted}[fontsize=\footnotesize]{smalltalk}
searchDepth: k forDLDataObject: r partialSelection: cont
  ^ self
      onEnter: [ r untilFixPointOf: [ :ro | ro right ] 
                   do: [ :j | j column cover ] ]
      do: [ self
              searchDepth: k + 1
              forDLRootObject: r column root
              partialSelection: [ :sel | 
                cont
                  value:
                    (ValueLink new
                      value: r model;
                      nextLink: sel;
                      yourself) ] ]
      onExit: [ r untilFixPointOf: [ :ro | ro left ] 
                  do: [ :j | j column uncover ] ]
\end{minted}
\end{frame}
  
\begin{frame}[fragile]
\frametitle{AlgorithmX instance side}
\begin{minted}[fontsize=\footnotesize]{smalltalk}
searchDLRootObject: h onSolutionDo: aBlock
  ^ self
      searchDepth: 0
      forDLRootObject: h
      partialSelection: [ :selLink | 
        aBlock value: (LinkedList new add: selLink; asSet) ]
\end{minted}
\vfill
Knuth advices to use the heuristic (provided by \texttt{DLRootObject} objs)
\begin{minted}[fontsize=\footnotesize]{smalltalk}
chooseColumn
  | c s |
  s := Float infinity.
  self
    untilFixPointOf: [ :ro | ro right ]
    do: [ :j | 
      j size < s
        ifTrue: [ c := j.
          s := j size ] ].
  ^ c
\end{minted}
that \textit{minimizes} the search tree's branching factor.
\end{frame}

\begin{frame}[fragile]
\frametitle{DLColumnObject instance side}
The operation of covering column $c$ is more interesting: 
It removes $c$ from the header list and removes all rows in $c$'s own list 
from the other column lists they are in.
\begin{minted}[fontsize=\footnotesize]{smalltalk}
cover
  we remove.
  self
    untilFixPointOf: [ :co | co down ]
    do: [ :i | 
      i
        untilFixPointOf: [ :do | do right ]
        do: [ :j | 
          j nsLink remove.
          j column updateSize: [ :s | s - 1 ] ] ]
\end{minted}
Operation $(1)$ is used here to remove objects in both the horizontal and vertical directions.
\end{frame}

\begin{frame}[fragile]
\frametitle{DLColumnObject instance side}
Finally, we get to the operation of \textit{uncovering} a given column $c$. 
Here is where the links do their dance:
\begin{minted}[fontsize=\footnotesize]{smalltalk}
uncover
  self
    untilFixPointOf: [ :co | co up ]
    do: [ :i | 
      i
        untilFixPointOf: [ :do | do left ]
        do: [ :j | 
          j nsLink restore.
          j column updateSize: [ :s | s + 1 ] ] ].
  we restore
\end{minted}
Notice that uncovering takes place in precisely the reverse order of the 
covering operation, using the fact that $(2)$ undoes $(1)$.
\end{frame}

\begin{frame}[fragile]
\frametitle{DLDataObject class side}
\begin{minted}[fontsize=\footnotesize]{smalltalk}
gridOn: aCollection
  | rootObj columns rows headers allObjs |
  aCollection
    sort: [ :vAssoc :wAssoc | 
      | v w |
      v := vAssoc key.
      w := wAssoc key.
      v y <= w y and: [ v x <= w x ] ].
  allObjs := Dictionary new.
  headers := DoubleLinkedList new.
  columns := Dictionary new.
  rows := Dictionary new.
  rootObj := DLRootObject new
    addInDoubleLinkedList: headers direction: #we;
    yourself.
  allObjs at: #root put: rootObj.
  "to be contd..."
\end{minted}
\end{frame}

\begin{frame}[fragile]
\frametitle{DLDataObject class side}
\begin{minted}[fontsize=\footnotesize]{smalltalk}
gridOn: aCollection
  "...contd..."
  aCollection
    do: [ :anAssociation | 
      | aPoint columnObj dataObj column row |
      aPoint := anAssociation key.
      column := columns
        at: aPoint y
        ifAbsentPut: [ | headerObj newColumn |
          headerObj := DLColumnObject new size: 0; root: rootObj; yourself.
          aPoint primary
            ifTrue: [ headerObj addInDoubleLinkedList: headers 
                                direction: #we ]
            ifFalse: [ DoubleLinkedList
                circular: [ :dll | headerObj addInDoubleLinkedList: dll 
                                             direction: #we ] ].
          newColumn := DoubleLinkedList new.
          headerObj addInDoubleLinkedList: newColumn direction: #ns.
          allObjs at: aPoint y put: headerObj.
          newColumn ].
  "..to be contd further..."
\end{minted}
\end{frame}

\begin{frame}[fragile]
\frametitle{DLDataObject class side}
\begin{minted}[fontsize=\footnotesize]{smalltalk}
gridOn: aCollection
  "...contd"
      columnObj := column first.
      dataObj := DLDataObject new
        column: columnObj;
        point: aPoint;
        model: anAssociation value;
        yourself.
      row := rows at: aPoint x ifAbsentPut: [ DoubleLinkedList new ].
      dataObj
        addInDoubleLinkedList: column direction: #ns;
        addInDoubleLinkedList: row direction: #we.
      columnObj updateSize: [ :s | s + 1 ].
      allObjs at: aPoint put: dataObj ].
  headers makeCircular.
  columns valuesDo: #makeCircular.
  rows valuesDo: #makeCircular.
  ^ allObjs
\end{minted}
\end{frame}


\begin{frame}{ }
\Huge Thanks!
\end{frame}

\begin{frame}[fragile]
\begin{minted}[fontsize=\footnotesize]{smalltalk}
yieldNode: tree onBlock: cont
  tree sets
    collect: [ :each | (each collect: #model) as: LinkedList ]
    thenDo: [ :sel | 
      | link |
      link := sel isEmpty ifTrue: [ nil ] ifFalse: [ sel firstLink ].
      cont value: link ].
  ^ tree
\end{minted}
\vfill
\begin{minted}[fontsize=\footnotesize]{smalltalk}
uniqueNodeWithDLDataObject: r withLowerNode: x withHigherNode: y
  | key |
  key := Array with: r with: x with: y.
  ^ zDDTree
    at: key
    ifAbsentPut: [ | z |
      z := ZDDNode new model: r; lower: x; higher: y; yourself.
      x parent: z.
      y parent: z.
      z ]
\end{minted}
\end{frame}


\end{document}

